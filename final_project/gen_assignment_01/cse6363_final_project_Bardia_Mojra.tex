\documentclass{homeworg}

\usepackage{authblk}
\title{CSE 5301 - HW02}
\author[1]{Bardia Mojra}
\affil[1]{1000766739}
\begin{document}

\maketitle


\exercise
We have a random variable Z which belongs to the binomial distribution. Suppose
the probability of the success is 0.2 and number of experiments is 12,  what is
the probability for each one of the below statements: (5 points each)
\newline
Binomial distribution:
$$
P_z = P(Z=z) = \binom{n}{z} p^z (1-p)^{n-z}~; ~~z \in \mathbb{Z} ~ (or~z \in \{\mathbb{N}, 0\})
$$
$$
P_z = \frac{n!}{z!(n-z)!} ~ .~ p^z ~.~(1-p)^{n-z}
$$
Where:
$$
p = 0.2, ~ n = 12
$$

\newline
a) P(Z=5)
\newline
$$
P(Z=5) = \frac{n!}{z!(n-z)!} ~ .~ p^z ~.~(1-p)^{n-z}
\Rightarrow ~ \frac{12!}{5!(12-5)!} ~ .~ (.2)^5 ~.~(.8)^{7} ~ = ~ 0.0532
$$
\newline
b) P(Z<3)
\newline
$$
P(Z<3) = P(0)+P(1)+P(2) =
\frac{12!~(.2)^0 ~(.8)^{12}}{0!(12)!}~+
~\frac{12!~(.2)^1 ~(.8)^{11}}{1!(11)!}~+
~\frac{12!~(.2)^2 ~(.8)^{10}}{2!(10)!} ;~0!\equiv 1
$$
\newline
$$
\Rightarrow .0687 + 0.2061 + .2835 = .5583
$$
\newline
c) P (Z>=7)
$$
P(Z \geqslant 7) = P(7)+P(8)+P(9)+P(10)+P(12)=
$$
\newline
$$
\frac{12!~(.2)^7 (.8)^{5}}{7!(5)!}+
\frac{12!~(.2)^8 (.8)^{4}}{8!(4)!}+
\frac{12!~(.2)^9 (.8)^{3}}{9!(3)!}+
\frac{12!~(.2)^{10} (.8)^{2}}{10!(2)!}+
\frac{12!~(.2)^{11} (.8)^{1}}{11!(1)!}+
\frac{12!~(.2)^{12} (.8)^{0}}{12!(0)!}
$$
\newline
$$
.0033+.0005+.000057672+.000004325+.000000197+.000000004
~ = ~ .003862198
$$

\exercise
Suppose that we investigated a fast food chain and saw that on average 6 out of
10 customers order drinks along with their meals. We choose 20 customers
randomly, find the probability that : (5 points each)
\newline
$$
P_z = \frac{n!~.~p^z~.~(1-p)^{n-z}}{z!(n-z)!};~n=20,~p=.6
$$
\newline
a) Exactly 10 customer order drinks with their meals.
\newline
$$
P_{10} = \frac{20!~.~(.6)^{10}~.~(.4)^{10}}{10!(10)!}
= ~ .117141551
$$
\newline
b) More than 15 customer order drinks with their meals.
$$
P_{>15} =
\frac{20!~.~(.6)^{16}~.~(.4)^{4}}{16!(4)!}+
\frac{20!~.~(.6)^{17}~.~(.4)^{3}}{17!(3)!}+
\frac{20!~.~(.6)^{18}~.~(.4)^{2}}{18!(2)!}+
\frac{20!~.~(.6)^{19}~.~(.4)^{1}}{19!(1)!}+
\frac{20!~.~(.6)^{20}~.~(.4)^{0}}{20!(0)!}
$$
\newline
$$
\Rightarrow~.0349+.01234+.00309+.00049+.000036~=~.050856
$$
\newline
c) What is the variance of the number of drinks?
\newline
$$
Var_{binomial}~=~n.p.(1-p) \Rightarrow 20\cdot.6\cdot.4~=~4.8
$$
\newline
d) Name the distribution
\newline
Binomial distribution.

\exercise
We have a deck of cards (52 cards) and we draw 6 cards randomly without
replacement. calculate the probability that: (5 points each)
\newline

a) We have 4 red cards.
\newline
$$
\frac{26}{52}\cdot\frac{25}{51}\cdot\frac{24}{50}\cdot\frac{23}{49}~=
~ .0552 = 5.52\%
$$
\newline
b) We have 3 face cards.
\newline
$$
\frac{12}{52}\cdot\frac{11}{51}\cdot\frac{10}{50}~=~.00995~=~.995\%
$$
\newline
c) Name the distribution.
\newline
Normal distribution.
\newline


\exercise
Suppose that we saw the power outage on average rate of 4 per month. What is
the probability that in the next 2 months we see : (5 points each)
$$
P(K=k) = e^{-\lambda}\cdot\frac{\lambda^k}{k!}~;~~ \lambda~=~4~per~month~=~8~per~2~months
$$
a) Exactly 8 power outage?
\newline
$$
P(k=8)=e^{-8}\cdot\frac{8^8}{8!}= .13958
$$
\newline
b) At most 4 power outage?
\newline
$$
P(k\leqslant 4)=P(0)+P(1)+P(2)+P(3)+P(4)
$$
\newline
$$
\Rightarrow
e^{-8}\cdot\frac{8^0}{0!}+
e^{-8}\cdot\frac{8^1}{1!}+
e^{-8}\cdot\frac{8^2}{2!}+
e^{-8}\cdot\frac{8^3}{3!}+
e^{-8}\cdot\frac{8^4}{4!}
=.00034+.00268+.01073+.02863+.05725 = .09963
$$
\newline
c) What is the expected value of the number of power outage?
\newline
$$
MLE_{Poisson} = \mu~[mean] = \lambda = 4~~per~~month~or~8~for~the~next~2~months
$$
\newline
d) Name the distribution
\newline
Poisson distribution.
\newline

\exercise
We are examining a product which was made by a company and we noticed that the
product is defective with the probability of \(20\%\).  Calculate the below
statements: (5 points each)
\newline
Geometric distribution:
$$
PDF_{Geometric}=P(Z=z)=q^{z-1} \cdot p;~~z\in \mathbb{N}
$$
$$
CDF_{Geometric}=P(Z\leqslant z)=1-q^{z} ;~~z\in \mathbb{N}
$$
\newline
a) The probability that the first defective product is the 4th one than we
examine.
\newline
$$
P(z=4)=(.8)^{3} \cdot .2 = .1024
$$
\newline
b) Calculate the probability that we find the first defective product in the
first 5 inspections?
\newline
$$
P(.)=P(1)+P(2)+P(3)+P(4)+P(5)= 1-(.8)^5=.67232
$$
\newline
c) Name the distribution.
\newline
Geometric distribution.

\exercise
We have a fair die which has 4 faces (1, 2, 3, 4) and our random variable X
shows the number when the die is rolled. (5 points each)
\newline
a) What is the probability mass function for the random variable x?
\newline
$$
P(X=x)=\frac{1}{4}
$$
b) What is the expected value of X?
\newline
$$
\mu = \frac{1+2+3+4}{4}=2.5
$$
\newline
c) Name the distribution
\newline
Uniform distribution. 

\bibliography{ref}

\end{document}
